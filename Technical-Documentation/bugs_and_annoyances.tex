The following are known bugs/problems with Galant. Some of them, marked
\textbf{*} have been observed only on a Mac (on my Mac specifically).

\subsubsection*{Input/output}

\begin{itemize}

\item
  If a graph has been saved via a \Code{File$\rightarrow$Export} in the graph
  window, runtime attributes such as \Code{hidden} may cause problems when
  the graph is read and displayed. The issue has been fixed for the
  particular case of \Code{hidden}, but not related ones such as
  \Code{hiddenLabel} and \Code{hiddenWeight}. Nonstandard attributes are
  saved as strings only even if the algorithm treated them as numbers or
  Booleans.

  \textbf{\emph{The best workaround when an exception occurs during reading
      of a \Code{graphml} file is to move the file to another
      location. Otherwise the exception will occur each time you open Galant.}}
  
\item \textbf{*} Occasionally there is an exception involving the user
  interface with a stack entry that says something about
  \Code{GlyphView}. This can happen at any time. Just hit \Code{Continue}
  when the error window pops up (usually a good policy if you're not sure
  what the error is about and it appears not to be affecting your session).
\end{itemize}

\subsubsection*{Text editing (of programs or graphs)\footnote{
Galant's editor is primitive, but
programs can easily be edited externally.
Text representations of graphs can either be edited or generated externally.}}

\begin{itemize}

\item There is no ``undo'' mechanism, either for the text editor or the
  editor in the graph window.

\item \textbf{*} Tabs for graphs and algorithms are often hard to deal with: (a) if you
  reread a graph or algorithm, it appears twice; (b) you can only run an
  algorithm on a graph if the tabs for the two appear at the top of the
  window at the same time -- this
  may be impossible if there are other intervening graphs and algorithms
  and the window is not wide enough.

\item \textbf{*} Occasionally Galant simply hangs;
  it appears that you can exit (quit) from
  the file menu of the graph window or from the the \Code{Galant} menu on the
  task bar.
  Oddly, when Galant hangs, you can bring it back to life by executing a
  command in a separate terminal window.

\item
  Some preferences, such as syntax highlight colors, require the user to exit
  Galant before they take effect.

\item \textbf{*}
  The \Code{Open/Save} preference that specifies the default directory for
  files does not persist from one session to the next (unlike all other preferences).

\item
  When saving a file, Galant complains if the extension is not correct
  (\textsf{.graphml} or \textsf{.alg}) but does not fill it in automatically.

\end{itemize}

\subsubsection*{Graph editing (in graph panel or via keyboard shortcuts)}

\begin{itemize}

\item Editing is mode-driven: the effect of a mouse action is determined by
  which of the four modes (select, create node, create edge, delete) is
  selected on the toolbar. This has unpleasant consequences if, for example,
  the user forgets that she is in delete mode.

\item Nodes have to be moved individually. In a large graph there is no way
  to select a collection of nodes and move them all at once.

\item The force directed layout algorithm clusters nodes too close to each
  other when there are cliques or near cliques.

\item Semantics of force directed layout when combined with adding edges is
  not intuitive (force directed layout takes over if the button is pressed,
  so adding nodes/edges is dependent on the state of the button). It's
  generally a bad idea to change the graph when the smart reposition button
  is pressed.

\item It is not possible to change the thickness of an edge or node boundary
  directly from the editor or an algorithm nor is it possible to change the
  fill color of a node.  The only way to change these properties is via
  highlighting, selecting, and marking nodes/edges during the animation.

\item Once you choose a color for a node in the editor, you can't uncolor it in
  the editor. The best you can do is set it to black, but then it appears
  thicker. However, the algorithm \Code{strip\_attributes.alg} can be run to
  reset colors and other nonessential attributes. You can save the cleaned up
  graph using \Code{File$\rightarrow$Export} in the graph window.

\item It is not possible to change a weight from nothing to 0. You need to
  set it to 1 first and then back to 0.

\item When user creates a new node/edge via keyboard shortcut, there is no
  obvious way to enter the weight and label (except to click in the
  appropriate text field).

\end{itemize}

\subsubsection*{Compilation and execution}

\begin{itemize}

\item \textbf{*}
  If you try to manipulate anything in the text window (e.g., click on
  \Code{File}) when an algorithm is running, Galant hangs. You must quit
  Galant completely without saving anything.
  
\item
  When there are compilation errors the user cannot scroll the text window
  (or make modifications in it) while the popup window showing the errors is
  displayed. There are two possible workarounds: (i)~open the algorithm in an
  external editor, or (ii)~view the error messages in the console.

\item Every once in a while an exception occurs when an error-free algorithm
  is executed or when Galant initially fires up, but it is possible to step
  through the animation normally after hitting the \textsf{Continue} button.

\item When an algorithm controls visibility of node/edge labels or weights
  and the user overrides in the middle of execution, Galant sometimes freezes
  when user terminates the algorithm. This appears to happen more frequently
  if user does a lot of fast forward/reverse between visibility changes. The
  problem does not seem to occur if user toggles visibility via keyboard
  shortcuts.

\item
  Compiler error messages can be cryptic (but at least they refer to the correct
  line numbers). Because of the macro preprocessing,
  it may be necessary to look at the console
  to get an idea of what is causing a particular error.
  If the parentheses/brackets/braces inside a function definition or body of
  one of the \Code{for\_\ldots} macros are unbalanced, the macro preprocessor
  will simply report the fact with no indication of the location of the error
  except for an excerpt from the beginning of the body.

\item Line numbers do, however, get out of sync if the header of a function
  declaration takes up more than one line. For example, in
\begin{verbatim}
     function foo(Node v,
                  Node w,
                  Edge e) {
     }
\end{verbatim}
The first three lines are treated as one.

\item
  If a macro is used incorrectly,
  the preprocessor does not report a line number.

\item After hitting \Code{Enter} or \Code{Return} at the end of a query,
  user still needs to step forward to do the next step of the algorithm.

\item
  There is no way to execute the animation in a continuous fashion with a
  controllable speed.
  The current workaround is the use of arrow keys as keyboard
  shortcuts for stepping forward or backward -- these can be held down to
  generate multiple steps in rapid succession, but finer grained control is
  difficult.

\end{itemize}

% [Last modified: 2017 07 22 at 13:38:29 GMT]
