\begin{table}
  \small
  \centering
  \begin{tabular}{| m{0.35\textwidth} | m{0.6\textwidth} |}
    \hline
    \textsf{print(String s)}
    &
    prints \texttt{s} on the console; useful for debugging
    \\ \hline
    \textsf{display(String s)}
    &
    writes the string \textsf{s} at the top of the window
    \\ \hline
    \textsf{String getMessage()}
    &
    returns the message currently displayed on the message banner
    \\ \hline
    \textsf{error(String s)}
    &
    prints \textsf{s} on the console with a stack trace; also displays
    \textsf{s} in popup window with an option to view the stack trace;
    the algorithm terminates and the user can choose whether to terminate
    Galant entirely or continue interacting
    \\ \hline
    \textsf{beginStep()},
    \textsf{endStep()}
    &
    any actions between a \textsf{beginStep()} and an \textsf{endStep()}
    take place atomically, i.e.,
    all in a single ``step forward'' action by the user
    \\ \hline
    \textsf{Node getNode(String message)}
    &
    pops up a window with the given message and prompts the user to enter the
    identifier of a node, which is returned;
    if no node with that id exists,
    an error popup is displayed and the user is prompted again
    \\ \hline
    \textsf{Edge getEdge(String message)}
    &
    pops up a window with the given message and prompts the user to enter the
    identifiers of two nodes, the endpoints of an edge, which is returned;
    if either id has no corresponding node or the the two nodes are not connect
    by an edge (in the right direction if the graph is directed),
    an error popup is displayed and the user is prompted again
    \\ \hline
    \shortstack[l] {
      \mbox{}
      \\[\smallskipamount]
      \textsf{Node getNode(String p,} \\
      \textsf{~~~~~~~~~~~~~~~~~~~NodeSet s,} \\
      \textsf{~~~~~~~~~~~~~~~~~~~String e)} \\
      \textsf{Edge getEdge(String p,} \\
      \textsf{~~~~~~~~~~~~~~~~~~~EdgeSet s,} \\
      \textsf{~~~~~~~~~~~~~~~~~~~String e)} \\
      \mbox{}
    }
    &
    variations of \textsf{getNode} and \textsf{getEdge}; here \textsf{p}
    is the prompt, \textsf{s} is the set from which the node or edge must be
    chosen and \textsf{e} an error message if the node/edge does not belong to the specified set;
    useful when wanting to specify an adjacent node or an outgoing edge
    \\ \hline
    \shortstack[l] {
      \mbox{}
      \\[\smallskipamount]
      \textsf{String getString(String message)}\\
      \textsf{Integer getInteger(String message)}\\
      \textsf{Double getReal(String message)} \\
      \mbox{}
    }
    &
    analogous to \textsf{getNode} and \textsf{getEdge}; allow algorithm to engage
    in dialog with the user
    \\ \hline
    \shortstack[l] {
      \mbox{}
      \\[\smallskipamount]
      \textsf{Integer integer(String s)}\\
      \textsf{Double real(String s)} \\
      \mbox{}
    }
    &
    performs conversion from a string to an integer/double; useful when parsing
    labels that represent numbers
    \\ \hline
    \textsf{windowWidth()}, \textsf{windowHeight()}
    &
    current width and height of the window, in case the algorithm wants to rescale
    the graph
    \\ \hline
  \end{tabular}
  \caption{Utility functions.}
  \label{tab:utility_functions}
\end{table}

