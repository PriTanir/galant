\begin{table}
  \small
  \centering
  \begin{tabular}{| m{0.37\textwidth} | m{0.58\textwidth} |}
    \hline
    \shortstack[l]{
      \textsf{Boolean nodeLabelsAreVisible()}\\
      \textsf{Boolean edgeLabelsAreVisible()}\\
      \textsf{Boolean nodeWeightsAreVisible()}\\
      \textsf{Boolean edgeWeightsAreVisible()}
    }
    &
    returns \textsf{true} if node/edge labels/weights are \emph{globally} visible,
    the default state, which can be altered by \textsf{hideNodeLabels()}, etc.,
    defined below
    \\ \hline
    \shortstack[l]{
      \textsf{hideNodeLabels(), hideEdgeLabels()}\\
      \textsf{hideNodeWeights(), hideEdgeWeights()}
    }
    &
    hides all node/edge labels/weights; typically used at the beginning of an algorithm
    to hide unnecessary information; labels and weights are shown by default
    \\ \hline
    \shortstack[l]{
      \textsf{showNodeLabels(), showEdgeLabels()}\\
      \textsf{showNodeWeights(), showEdgeWeights()}
    }
    &
    undoes the hiding of labels/weights
    \\ \hline
    \shortstack[l]{
      \textsf{hideAllNodeLabels()}\\
      \textsf{hideAllEdgeLabels()}\\
      \textsf{hideAllNodeWeights()}\\
      \textsf{hideAllEdgeWeights()}
    }
    &
    hides all node/edge labels/weights even if they are visible globally
    by default or by
    \textsf{showNodeLabels()}, etc., or for individual nodes and edges;
    in order for the label or weight of a node/edge to be displayed,
    labels/weights must be visible globally and its label/weight must be visible;
    initially, all labels/weights are visible, both globally and for individual
    nodes/edges;
    these functions are used to hide information so that it can be revealed subsequently,
    one node or edge at a time
    \\ \hline
    \shortstack[l]{
      \textsf{showAllNodeLabels()}\\
      \textsf{showAllEdgeLabels()}\\
      \textsf{showAllNodeWeights()}\\
      \textsf{showAllEdgeWeights()}
    }
    &
    makes all individual
    node/edge weights/labels visible if they are globally visible by default
    or via \textsf{showNodeLabels()}, etc.;
    this undoes the effect of \textsf{hideAllNodeLabels()}, etc., and of
    any individual hiding of labels/weights
    \\ \hline
    \shortstack[l]{
      \textsf{clearNodeLabels(), clearEdgeLabels()}\\
      \textsf{clearNodeWeights(), clearEdgeWeights()}
    }
    &
    gets rid of all node/edge labels/weights; this not only makes them invisible,
    but also erases whatever values they have
    \\ \hline
    \textsf{showNodes(), showEdges()}
    &
    undo any hiding of nodes/edges that has taken place during the algorithm
    \\ \hline
    \shortstack[l]{
      \Code{NodeSet visibleNodes()} \\
      \Code{EdgeSet visibleEdges()}
      }
    &
    return the set of nodes/edges that are not hidden
    \\ \hline
    \shortstack[l]{
      \textsf{clearNodeMarks()}\\
      \textsf{clearNodeHighlighting()}\\
      \textsf{clearEdgeHighlighting()}
    }
    &
    unmarks all nodes, unhighlights all nodes/edges, respectively
    \\ \hline
    \shortstack[l]{
      \textsf{clearNodeLabels()}\\
      \textsf{clearNodeWeights()}\\
      \textsf{clearEdgeLabels()}\\
      \textsf{clearEdgeWeights()}
    }
    &
    erases labels/weights of all nodes/edges; useful if an algorithm needs to
    start with a clean slate with respect to any of these attributes
    \\ \hline
    \shortstack[l]{
      \textsf{clearAllNode(String attribute)}\\
      \textsf{clearAllEdge(String attribute)}
    }
    &
    erases values of the given attribute from all nodes/edges, a generalization of
    \textsf{clearNodeLabels}, etc.
    \\ \hline
    \textsf{boolean set(String attribute, $\langle$\emph{type}$\rangle$ value)}
    &
    sets an arbitrary attribute of the graph to have a value of a given type, where
    the type is one of \textsf{Integer}, \textsf{Double}, \textsf{Boolean}
    or \textsf{String};
    in the special case of \textsf{Boolean} the second argument may be omitted
    and defaults to \textsf{true};
    so \textsf{set("attr")} is equivalent to \textsf{set("attr",true)};
    returns \textsf{true} if the graph already has a value for the given attribute,
    \textsf{false} otherwise
    \\ \hline
    \shortstack[l]{
    \textsf{$\langle$\emph{type}$\rangle$ get$\langle$\emph{type}$\rangle$(String attribute)}\\
    \textsf{Boolean is(String attribute)}
    }
    &
    returns the value associated with \textsf{attribute} or \textsf{null}
    if the graph has no value of the given type for \textsf{attribute}, i.e.,
    if no
    \textsf{set(String~attribute,~$\langle$\emph{type}$\rangle$~value)} has occurred;
    in the special case of a \textsf{Boolean} attribute, the second formulation
    may be used
    \\ \hline
    \shortstack[l]{
      \textsf{clearAllNode(String attribute)}\\
      \textsf{clearAllEdge(String attribute)}
    }
    &
    erases the value of the given attribute for all nodes/edges
    \\ \hline
  \end{tabular}

  \caption{Functions that query and manipulate graph
    node and edge attributes globally, i.e., for all nodes or edges
    at once. Also included are functions that deal with graph attributes.}
  \label{tab:graph_attribute_functions}
\end{table}

% [Last modified: 2016 12 20 at 22:26:17 GMT]
