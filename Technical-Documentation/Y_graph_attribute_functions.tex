\begin{table}
\centering

\small
\parbox{0.9\textwidth}{
  These functions are designed to access or manipulate attributes for all nodes or edges
  at once instead of individually.
  Also included are functions that deal with graph attributes.
}

\medskip
  \begin{tabular}{| m{0.37\textwidth} | m{0.58\textwidth} |}
    \hline
    \shortstack[l]{
      \Code{hideAllNodeLabels()}\\
      \Code{hideAllEdgeLabels()}\\
      \Code{hideAllNodeWeights()}\\
      \Code{hideAllEdgeWeights()}
    }
    &
    hide all node/edge labels/weights
    these functions are typically
    used to hide information so that it can be revealed subsequently,
    one node or edge at a time
    \\ \hline
    \shortstack[l]{
      \Code{showAllNodeLabels()}\\
      \Code{showAllEdgeLabels()}\\
      \Code{showAllNodeWeights()}\\
      \Code{showAllEdgeWeights()}
    }
    &
    make all individual
    node/edge weights/labels visible
    this undoes the effect of \Code{hideAllNodeLabels()}, etc., and of
    any individual hiding of labels/weights
    \\ \hline
    \shortstack[l]{
      \Code{clearNodeLabels(), clearEdgeLabels()}\\
      \Code{clearNodeWeights(), clearEdgeWeights()}
    }
    &
    gets rid of all node/edge labels/weights; this not only makes them invisible,
    but also erases whatever values they have
    \\ \hline
    \Code{showNodes(), showEdges(), showGraph()}
    &
    undo any hiding of nodes/edges that has taken place during the algorithm;
    \Code{showNodes()} only shows edges currently (logically) visible; \Code{showGraph()}
    restores visibility of both nodes and edges
    \\ \hline
    \shortstack[l]{
      \Code{NodeSet visibleNodes()} \\
      \Code{EdgeSet visibleEdges()}
      }
    &
    return the set of nodes/edges that are not hidden
    \\ \hline
    \shortstack[l]{
      \Code{clearNodeMarks()}\\
      \Code{clearNodeHighlighting()}\\
      \Code{clearEdgeHighlighting()}
    }
    &
    unmarks all nodes, unhighlights all nodes/edges, respectively
    \\ \hline
    \shortstack[l]{
      \Code{clearNodeLabels()}\\
      \Code{clearNodeWeights()}\\
      \Code{clearEdgeLabels()}\\
      \Code{clearEdgeWeights()}
    }
    &
    erases labels/weights of all nodes/edges; useful if an algorithm needs to
    start with a clean slate with respect to any of these attributes
    \\ \hline
    \shortstack[l]{
      \Code{clearAllNode(String attribute)}\\
      \Code{clearAllEdge(String attribute)}
    }
    &
    erases values of the given attribute from all nodes/edges, a generalization of
    \Code{clearNodeLabels}, etc.
    \\ \hline
    \Code{boolean set(String attribute, $\langle$\emph{type}$\rangle$ value)}
    &
    sets an arbitrary attribute of the graph to have a value of a given type, where
    the type is one of \Code{Integer}, \Code{Double}, \Code{Boolean}
    or \Code{String};
    in the special case of \Code{Boolean} the second argument may be omitted
    and defaults to \Code{true};
    so \Code{set("attr")} is equivalent to \Code{set("attr",true)};
    returns \Code{true} if the graph already has a value for the given attribute,
    \Code{false} otherwise
    \\ \hline
    \shortstack[l]{
    \Code{$\langle$\emph{type}$\rangle$ get$\langle$\emph{type}$\rangle$(String attribute)}\\
    \Code{Boolean is(String attribute)}
    }
    &
    returns the value associated with \Code{attribute} or \Code{null}
    if the graph has no value of the given type for \Code{attribute}, i.e.,
    if no
    \Code{set(String~attribute,~$\langle$\emph{type}$\rangle$~value)} has occurred;
    in the special case of a \Code{Boolean} attribute, the second formulation
    may be used
    \\ \hline
    \shortstack[l]{
      \Code{clearAllNode(String attribute)}\\
      \Code{clearAllEdge(String attribute)}
    }
    &
    erases the value of the given attribute for all nodes/edges
    \\ \hline
  \end{tabular}

  \caption{Functions that query and manipulate graph
    node and edge attributes globally.}
  \label{tab:graph_attribute_functions}
\end{table}

% [Last modified: 2017 01 23 at 19:26:54 GMT]
