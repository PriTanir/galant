Some features under development for the next major release are listed here.

\begin{enumerate}
\item
  \emph{Mode-less graph editing.}
  In place of the GDR-like mechanism, where panel buttons are used to determine
  how the graph editor responds to mouse actions -- a click might create a node,
  initiate an edge or select an object --
  the \textsf{Ctrl} and \textsf{Shift} keys can determine the ``mode''.
  So, for example, \textsf{Ctrl-Click} would create a node and
  \textsf{Shift-Click} would initiate an edge.

\item \emph{Scrolling in the graph window.} Currently, if the graph is too
  large to fit in the current window, the only recourse is to do
  force-directed layout.

\item \emph{Scaling of graph within window.} A possible alternative or option
  to the previous item is making the node positions adjust so that they
  always fit into the current window. How best to handle the semantics is not
  clear: you don't necessarily want the editor to change x and y attributes
  in the GraphML whenever user resizes the window. It may make sense to
  establish a grid based on maximum horizontal and vertical dimensions when
  the graph is first loaded.

\item
  \emph{Mapping attributes to actions.}
  In order to make animations more accessible to visually impaired users,
  there should be a mechanism that, under user control, specifies how Boolean
  attributes such as marking or highlighting are ``displayed''.
  Currently, the thickness of highlighted node borders and edges can be
  controlled in the \textsf{Preferences} panel.
  A more sophisticated mapping mechanism that incorporates sound as well as visuals is needed.
  The ultimate approach would allow mappings for arbitrary attributes
  defined by user or programmer.

\item \emph{Inflection points on edges.}  For animation of automata it's
  important to have curved edges if there is a transition going from state
  $q$ to state $r$ and another from $r$ to $q$; other applications may need
  this as well. A single inflection point, carefully placed, and present only
  if there are parallel edges, could accomplish this.

\item \emph{More preferences.} Font sizes for labels and weights, thickness for colors
  distinct from highlights, distinct highlight colors for Galant types versus functions.

\item \emph{Selection of multiple nodes and/or edges.}  It might prove useful
  to move a collection of nodes instead of just a single node or give a
  collection of nodes or edges the same color, label or weight. This is
  mostly for edit mode.
\end{enumerate}

% [Last modified: 2017 01 23 at 20:46:29 GMT]
