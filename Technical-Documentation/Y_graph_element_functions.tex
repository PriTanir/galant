\begin{table}
  \small
    Here, \emph{element} refers to either a \Code{Node} or an \Code{Edge},
    both as a type and as a formal parameter.

    \medskip
  \centering
  \begin{tabular}{| m{0.4\textwidth} | m{0.55\textwidth} |}
    \hline
    \Code{id(\emph{element})}
    &
    returns the unique identifier of the node or edge
    \\ \hline
    \Code{source(Edge e)}, \Code{target(Edge e)}
    &
    returns the source/target of edge \Code{e}, sometimes called the (arrow)
    tail/head or source/destination
    \\ \hline
    \Code{string(Edge e)}
    &
    returns a string of the form \Code{"($s$,$t$)"}, where $s =$
    \Code{source(e)} and $t =$ \Code{target(e)}
    \\ \hline
    \Code{mark(Node v), unmark(Node v)}
    &
    shades the interior of a node or undoes that
    \\ \hline
    \Code{Boolean marked(Node v)}
    &
    returns \Code{true} if the node is marked
    \\ \hline
    \Code{NodeList unmarkedNeighbors(Node v)}
    &
    returns a list of the adjacent nodes that are not marked
    \\ \hline
    \Code{highlight(\emph{element}), unhighlight(\emph{element})}
    &
    makes the node or edge highlighted, i.e.,
    thickens the border or line and makes it red / undoes the highlighting
    \\ \hline
    \Code{Boolean highlighted(\emph{element})}
    &
    returns \Code{true} if the node or edge is highlighted
    \\ \hline
    \shortstack[l]{
      \Code{select(\emph{element})}, \Code{deselect(\emph{element})}\\
      \Code{selected(\emph{element})}
    }
    &
    synonyms for \Code{highlight}, \Code{unhighlight} and \Code{highlighted}
    \\ \hline
    \shortstack[l]{
      \Code{Double weight(\emph{element})}\\
      \Code{setWeight(\emph{element}, double weight)}
    }
    &
    get/set the weight of the element
    \\ \hline
    \shortstack[l]{
      \Code{showWeight(\emph{element})},
      \Code{hideWeight(\emph{element})}\\
      \Code{Boolean weightIsVisible(\emph{element})}\\
      \Code{Boolean weightIsHidden(\emph{element})}
    }
    &
    make the weight of the element visible/invisible,
    query their visibility; weights of the element type
    have to be
    globally visible -- see Table~\ref{tab:graph_attribute_functions}
    -- for \Code{showWeight} to have an effect
    \\ \hline
    \shortstack[l]{
      \Code{String label(\emph{element})}\\
      \Code{label(\emph{element}, Object obj)}
    }
    &
    get/set the label of the element, the \Code{Object} argument allows an object
    of any other type to be converted to a (\Code{String}) label,
    as long as there is a \Code{toString} method, which is true of all major classes
    (you have to be careful, for example, to use \Code{Integer} instead of \Code{int})
    \\ \hline
    \shortstack[l]{
      \Code{showLabel(\emph{element})},
      \Code{hideLabel(\emph{element})}\\
      \Code{Boolean labelIsVisible(\emph{element})}\\
      \Code{Boolean labelIsHidden(\emph{element})}
    }
    &
    analogous to the corresponding weight functions
    \\ \hline
    \shortstack[l]{
      \Code{hide(\emph{element})},
      \Code{show(\emph{element})}\\
      \Code{Boolean hidden(\emph{element})}\\
      \Code{Boolean visible(\emph{element})}
    }
    &
    makes nodes/edges disappear/reappear and tests whether they are visible
    or hidden; useful when an algorithm (logically) deletes objects, but they
    need to be revealed again upon completion; if node is hidden, all its
    incident edges are hidden as well; \Code{show($v$)}, where $v$ is a node,
    will show the incident edges \emph{even if they were independently hidden}
    \\ \hline
     \shortstack[l]{
      \Code{String color(\emph{element})}\\
      \Code{color(\emph{element}, String c)}\\
      \Code{uncolor(\emph{element})}
    }
    &
    get/set/remove the color of the border of a node or line representing an edge;
    colors are encoded as strings of the form
    \Code{"\#RRBBGG"}, the RR, BB and GG being hexadecimal numbers representing the
    red, blue and green components of the color, respectively; see Table~\ref{tab:colors}
    for a list of predefined colors;
    when an element has no color, the line is thinner and black
    \\ \hline
    \Code{boolean set(\emph{element}, String key, $\langle$\emph{type}$\rangle$ value)}
    &
    sets an arbitrary attribute, \Code{key}, of the element to have a value of a given type, where
    the type is one of \Code{Integer}, \Code{Double}, \Code{Boolean}
    or \Code{String};
    in the special case of \Code{Boolean} the third argument may be omitted
    and defaults to \Code{true};
    so \Code{set(v,"attr")} is equivalent to \Code{set(v,"attr",true)};
    returns \Code{true} if the element already has a value for the given attribute,
    \Code{false} otherwise
    \\ \hline
    \Code{boolean clear(\emph{element}, String key)}
    &
    removes the attribute \Code{key} from the element; if the \Code{key} refers to
    a Boolean attribute, this is logically equivalent to making it false
    \\ \hline
    \shortstack[l]{
    \Code{$\langle$\emph{type}$\rangle$ get$\langle$\emph{type}$\rangle$(\emph{element}, String key)}\\
    \Code{Boolean is(\emph{element}, String key)}
    }
    &
    returns the value associated with \Code{key} or \Code{null}
    if the graph has no value of the given type for \Code{key}, i.e.,
    if no
    \Code{set(String~key,~$\langle$\emph{type}$\rangle$~value)} has occurred;
    in the special case of a \Code{Boolean} attribute, the second formulation
    may be used;
    the object-oriented syntax, such as \Code{e.is("inTree")}, sometimes
    reads more naturally
    \\ \hline
  \end{tabular}

  \caption{Functions that query and manipulate attributes of individual
    nodes and edges.
  }
  \label{tab:graph_element_functions}
\end{table}

% [Last modified: 2017 01 18 at 21:20:56 GMT]
