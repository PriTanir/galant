\begin{table}
  \small
  \textbf{NodeList and EdgeList:} lists of nodes or edges, respectively.
  We use \emph{list} as shorthand for either \textsf{NodeList~L} or
  \textsf{EdgeList L}, \emph{type} for either \textsf{Node} or \textsf{Edge}
  and \emph{element} for \textsf{Node v} or \textsf{Edge e}.
  
  \medskip
  \begin{tabular}{| m{0.35\textwidth} | m{0.6\textwidth} |}
    \hline
    \textsf{\emph{type} first(\emph{list})}
    &
    returns the first element of this list
    \\ \hline
    \textsf{add(\emph{element}, \emph{list})}
    &
    adds the element
    to the end of the list; along with \textsf{first}
    we get the effect of a queue
    \\ \hline
    \textsf{remove(\emph{element}, \emph{list})}
    &
    removes the first occurrence of the element from the list
    \\ \hline
    \textsf{sort(\emph{list})}
    &
    use the weights of the nodes/edges to sort the list \textsf{L};
    sort is actually a macro that invokes Java \textsf{Collections.sort(L)};
    this means that \textsf{L} can be any collection (list, stack, queue, set)
    of comparable elements
    \\ \hline
  \end{tabular}  
  
  \bigskip
  \textbf{NodeQueue and EdgeQueue:} queues of nodes or edges, respectively.
  Same coventions as for lists.

  \medskip
  \begin{tabular}{| m{0.35\textwidth} | m{0.6\textwidth} |}
    \hline
      \textsf{void~enqueue(\emph{element})}
    &
    adds the element to the rear of the queue
    \\ \hline
    \textsf{\emph{type} dequeue()}
    &
    returns and removes the element at the front of the queue;
    returns \textsf{null} if the queue is empty
    \\ \hline
    \textsf{\emph{type} remove()}
    &
    returns and removes the element at the front of the queue;
    throws an exception if the queue is empty
    \\ \hline
    \textsf{\emph{type}~element()}
    &
    returns the element at the front of the queue
    without removing it;
    throws an exception if the queue is empty
    \\ \hline
    \textsf{\emph{type} peek()}
    &
    returns the element at the front of the queue
    without removing it;
    returns \textsf{null} if the queue is empty
    \\ \hline
    \textsf{size()}
    &
    returns the number of elements in the queue
    \\ \hline
    \textsf{isEmpty()}
    &
    returns \textsf{true} if the queue is empty 
    \\ \hline
  \end{tabular}

  \bigskip
  \textbf{NodeStack and EdgeStack:} stacks of nodes or edges, respectively.

  \medskip
  \begin{tabular}{| m{0.35\textwidth} | m{0.6\textwidth} |}
    \hline
      \textsf{void~push(\emph{element})}
    &
    adds the element to the top of the stack
    \\ \hline
    \textsf{\emph{type}~pop()}
    &
    returns and removes the element at the top of the stack;
    throws an exception if the stack is empty
    \\ \hline
    \textsf{\emph{type} peek()}
    &
    returns the element at the top of the stack
    without removing it;
    returns null if the stack is empty
    \\ \hline
    \textsf{size()}, \textsf{isEmpty()}
    &
    analogous to the corresponding queue methods
    \\ \hline
  \end{tabular}

  \bigskip
  \textbf{NodePriorityQueue and EdgePriorityQueue:} priority queues of nodes or edges, respectively.

  \medskip
  \begin{tabular}{| m{0.35\textwidth} | m{0.6\textwidth} |}
    \hline
      \textsf{void~add(\emph{element})}
    &
    adds the element to the priority queue;
    the priority is defined to be its weight -- see Section~\ref{sec:display_attributes}
    \\ \hline
    \textsf{\emph{type}~removeMin()}
    &
    returns and removes the element with minimum weight;
    returns \textsf{null} if the queue is empty
    \\ \hline
    \hline
    \textsf{boolean~remove(\emph{element})}
    &
    removes the element;
    returns \textsf{true} if it is present,
    \textsf{false} otherwise
    \\ \hline
    \textsf{void~decreaseKey(\emph{element},~double~key)}
    &
    changes the weight of the element to the new key
    and reorganizes the priority queue appropriately;
    since this is accomplished by removing and reinserting the object, i.e.,
    inefficiently, this method can also be used to increase the key
    \\ \hline
    \textsf{size()}, \textsf{isEmpty()}
    &
    analogous to the corresponding queue methods
    \\ \hline
  \end{tabular}

  \bigskip
  \textbf{NodeSet and EdgeSet:} sets of nodes or edges, respectively.

  \medskip
  \begin{tabular}{| m{0.35\textwidth} | m{0.6\textwidth} |}
    \hline
    \textsf{boolean~add(\emph{element})}
    &
    adds the element to the set;
    returns true if the element was a new addition, false otherwise
    \\ \hline
    \textsf{boolean~remove(\emph{element})}
    &
    removes the element
    returns \textsf{true} if the element was present,
    \textsf{false} otherwise
    \\ \hline
    \textsf{boolean~contains(\emph{element})}
    &
    returns true if the element is in the set
    \\ \hline
    \textsf{size()}, \textsf{isEmpty()}
    &
    analogous to the corresponding methods for other data structures
    \\ \hline
  \end{tabular}

  \caption{Built-in data structures and their methods.
     These methods use object-oriented syntax:
$\langle$\emph{structure}$\rangle$.$\langle$\emph{method}$\rangle$($\langle$\emph{arguments}$\rangle$)
    and are created using, e.g.,
    \textsf{NodeQueue~Q~=~new~NodeQueue();} the \textsf{new} operator in Java.
  }
  \label{tab:data_structures}
\end{table}
