\begin{table}
  \small
  \centering

  \textbf{NodeList and EdgeList:} lists of nodes or edges, respectively.
  
  \medskip
  \begin{tabular}{| m{0.35\textwidth} | m{0.6\textwidth} |}
    \hline
    \shortstack[l]{
      \textsf{Node first(NodeList L)}\\
      \textsf{Edge first(EdgeList L)}
    }
    &
    returns the first element of this list
    \\ \hline
    \shortstack[l]{
      \textsf{add(Node v, NodeList L)}\\
      \textsf{add(Edge e, EdgeList L)}
    }
    &
    adds the node/edge to the end of the list \textsf{L}; along with \textsf{first}
    we get the effect of a queue
    \\ \hline
    \shortstack[l]{
      \textsf{remove(Node v, NodeList L)}\\
      \textsf{remove(Edge e, EdgeList L)}
    }
    &
    removes the first occurrence of node \textsf{v} or edge \textsf{e} from \textsf{L}
    \\ \hline
    \textsf{sort(NodeList L)}, \textsf{sort(EdgeList L)}
    &
    use the weights of the nodes/edges to sort the list \textsf{L}
    \\ \hline
  \end{tabular}  
  
\bigskip
  \textbf{NodeQueue and EdgeQueue:} queues of nodes or edges, respectively.

  \medskip
  \begin{tabular}{| m{0.35\textwidth} | m{0.6\textwidth} |}
    \hline
    \shortstack[l]{
      \textsf{void~enqueue(Node~v)},\\
      \textsf{void~enqueue(Edge~e)}
    }
    &
    adds \textsf{v} or \textsf{e} to the rear of the queue
    \\ \hline
    \textsf{Node~dequeue()}, \textsf{Edge~dequeue()}
    &
    returns and removes the \textsf{Node} or \textsf{Edge} at the front of the queue;
    returns \textsf{null} if the queue is empty
    \\ \hline
    \textsf{Node~remove()}, \textsf{Edge~remove()}
    &
    returns and removes the \textsf{Node} or \textsf{Edge} at the front of the queue;
    throws an exception if the queue is empty
    \\ \hline
    \textsf{Node~element()}, \textsf{Edge~element()}
    &
    returns the \textsf{Node} or \textsf{Edge} at the front of the queue
    without removing it;
    throws an exception if the queue is empty
    \\ \hline
    \textsf{Node~peek()}, \textsf{Edge~peek()}
    &
    returns the \textsf{Node} or \textsf{Edge} at the front of the queue
    without removing it;
    returns \textsf{null} if the queue is empty
    \\ \hline
    \textsf{size()}
    &
    returns the number of elements in the queue
    \\ \hline
    \textsf{isEmpty()}
    &
    returns \textsf{true} if the queue is empty 
    \\ \hline
  \end{tabular}

  \bigskip
  \textbf{NodeStack and EdgeStack:} stacks of nodes or edges, respectively.

  \medskip
  \begin{tabular}{| m{0.35\textwidth} | m{0.6\textwidth} |}
    \hline
    \shortstack[l]{
      \textsf{void~push(Node~v)},\\
      \textsf{void~push(Edge~e)}
    }
    &
    adds \textsf{v} or \textsf{e} to the top of the stack
    \\ \hline
    \textsf{Node~pop()}, \textsf{Edge~pop()}
    &
    returns and removes the \textsf{Node} or \textsf{Edge} at the top of the stack;
    throws an exception if the stack is empty
    \\ \hline
    \textsf{Node~peek()}, \textsf{Edge~peek()}
    &
    returns the \textsf{Node} or \textsf{Edge} at the top of the stack
    without removing it;
    returns null if the stack is empty
    \\ \hline
    \textsf{size()}, \textsf{isEmpty()}
    &
    analogous to the corresponding queue methods
    \\ \hline
  \end{tabular}

  \bigskip
  \textbf{NodePriorityQueue and EdgePriorityQueue:} priority queues of nodes or edges, respectively.

  \medskip
  \begin{tabular}{| m{0.35\textwidth} | m{0.6\textwidth} |}
    \hline
    \shortstack[l]{
      \textsf{void~add(Node~v)},\\
      \textsf{void~add(Edge~e)}
    }
    &
    adds \textsf{v} or \textsf{e} to the priority queue;
    the priority is defined to be its weight -- see Section~\ref{sec:display_attributes}
    \\ \hline
    \textsf{Node~removeMin()}, \textsf{Edge~removeMin()}
    &
    returns and removes the \textsf{Node} or \textsf{Edge} with minimum weight;
    returns \textsf{null} if the queue is empty
    \\ \hline
    \hline
    \shortstack[l]{
      \textsf{boolean~remove(Node~v)},\\
      \textsf{boolean~remove(Edge~e)}
    }
    &
    removes \textsf{v} or \textsf{e};
    returns \textsf{true} if the \textsf{v} or \textsf{e} is present,
    \textsf{false} otherwise
    \\ \hline
     \shortstack[l]{
      \textsf{void~decreaseKey(Node~v,~double~key)},\\
      \textsf{void~decreaseKay(Edge~e,~double~key)}
    }
    &
    changes the weight of \textsf{v} or \textsf{e} to the new key
    and reorganizes the priority queue appropriately;
    since this is accomplished by removing and reinserting the object, i.e.,
    inefficiently, this method can also be used to increase the key
    \\ \hline
    \textsf{size()}, \textsf{isEmpty()}
    &
    analogous to the corresponding queue methods
    \\ \hline
  \end{tabular}

  \caption{Built-in data structures and their methods.
     These methods use object-oriented syntax:
$\langle$\emph{structure}$\rangle$.$\langle$\emph{method}$\rangle$($\langle$\emph{arguments}$\rangle$)
    and are created using, e.g.,
    \textsf{NodeQueue~Q~=~new~NodeQueue();} the \textsf{new} operator in Java.
  }
  \label{tab:data_structures}
\end{table}
