There is a large variety of algorithm animations available. Here we focus solely on those that offer significant mechanisms for creating new animations.
These differ primarily in (i)~whether the user acts primarily as an observer; and
(ii)~the challenges imposed on the animator.
Almost all of these are systems rather than tools.
For a comprehensive survey of graph algorithm animation and graph drawing used
as educational tools see Bridgeman~\cite{2013-GDBook-Bridgeman}. 

\textbf{Observer-oriented (passive) systems.}
One general purpose animation program is ANIMAL~\cite{2002-JVLC-Roessling,ANIMAL};
it provides the animator
with a rich menu
of elements common to many algorithms.
Steps in the animation are linked to steps in the pseudocode.
Though there are many options for creating interesting animations,
it appears that these are passive.

Galles~\cite{Galles} is an animation tool with very
sophisticated creation options.
Primarily designed to be passive, it could conceivably, with a parser
for a graph input format and a mechanism that allows the user to view and
manipulate the input graph, be made interactive.
In that sense it would also be a tool.
It suffers, however, from the fact that the
animator must navigate a complex Java-based interface.

Although secondary to its main purpose as a library of data structures and
algorithms,
LEDA~\cite{1999-LEDA-Mehlhorn} offers a graph window facility that can be
used to create animations of graph algorithms.
The documentation gives several examples and illustrates the rich functionality of
the drawing and visualization capability of graph windows.
Since LEDA is a general purpose, C++-based, programming language for
algorithms and data structures, it is easily augmented with extensions that
are integrated seamlessly with the core API; in this case, graph windows work
in concert with core graph functions and macros.
%%  (which are similar to, but
%% more sophisticated than, those in GDR).
Unfortunately LEDA is a commercial product with
non-trivial licensing cost.

\textbf{Explorer-oriented (interactive) systems.}
Several online applets feature graph algorithm animation. Of these,
Javenga~\cite{JAVENGA} stands out. It is highly interactive. The drawing and
editing of
graphs is simple and intuitive, and graphs can be viewed in all three major
representations (drawing, adjacency matrix and adjacency list).
The variety of graph algorithms available is impressive:
breadth-first and depth-first search, topological sort, strongly connected
components, four shortest path algorithms, two minimum spanning tree
algorithms, and a network flow algorithm.
Animations can be run one step at a time with the option of moving backwards
or continuously with an adjustable number of milliseconds per step.
Javenga's main drawback is that the explorer is unable to save graphs for
future sessions.

The j-Alg~\cite{j-Algo} is an impressive animation system.
It is highly interactive, has a relatively easy to use interface, and has sophisticated animations for a large variety of algorithms,
including graph searching, Dijkstra's algorithm,
algebraic path problems (generalizations of all-pairs shortest paths and transitive closure), AVL trees, Knuth-Morris-Pratt string searching and BNF syntax diagrams.
Its only drawback is that there is no readily available mechanism
for outsiders to create new animations.
New animations are developer-created and released periodically.

\textbf{Libraries.}
AlgoViz~\cite{AlgoViz}
is a large catalog of algorithm animations, continually updated by
contributors who either submit new animations or comment on existing ones.
Like any large repository with many contributors, AlgoViz is difficult to
monitor and
maintain.
The OpenDSA project~\cite{%
2011-ProgramVisualization-Shaffer,2011-Koli-Shaffer,2012-SIGCSE-Fouh%
}
aims to create a textbook compilation of a variety of
visualizations, mostly designed for observers.

\textbf{Early work.}
Earlier animation tools/systems include GDR~\cite{1992-CSDM-Stallmann}, John Stasko's
Tango~\cite{1990-Computer-Stasko}, Xtango,
and SAMBA,\footnote{
These and Stasko's other animation tools are posted at
http://www.cc.gatech.edu/gvu/ii/softvis/
} and, of course,
the work of
Brown and Sedgewick~\cite{1988-Computer-Brown,1985-IEEE_Software-Brown}
and that of Bentley and Kernighan~\cite{1987-Animation-Bentley}.
SAMBA, and to a lesser extent GDR, is especially notable for emphasis on simplifying the creation of
animations so that students can easily accomplish them.
Both are also tools by our definition.
All of these suffer, however, from using old technology,
and, except for GDR,
they require off-line creation
of problem instances and have no graph-algorithm specific implementations
or graph creation interfaces to offer.

% [Last modified: 2013 06 26 at 18:34:44 GMT]
