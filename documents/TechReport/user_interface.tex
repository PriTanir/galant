Two windows appear when Galant is started: a \emph{graph window} shows the
current graph and a \emph{text window} shows editable text.
Depending on the currently selected tab in the text window,
the text can be either a GraphML description of the current graph or an
algorithm implementation. 

The user interface is designed for all three roles.
The observer (or an instructor demonstrating an algorithm) does as follows: (a)~loads a graph using the file browser; (b)~loads an algorithm; (c)~pushes the ``Compile and Run'' button; and (d)~uses
the controls underneath the graph window to step through the algorithm
forward or backward as desired.

A typical explorer might edit or create a graph using the graph window and then
follow steps (b)--(d), repeating steps (a) and (d) to try out different
graphs. Saving graphs for later use is also an option.
In addition, the explorer can use the graph's tab
in the text window to fine tune the placement of
nodes or apply a force-directed graph drawing method
(as described by Hu~\cite{2006-Mathematica-Hu}) to adjust node placement.

A creator can load and edit an existing algorithm or create one from scratch
using an appropriate tab in the text window.
Compilation and execution is accomplished via the buttons at the bottom.
In fact, the code of an animation is essentially a Java program with
(a)~predefined types for nodes and edges; (b)~an
API that interacts with the graph and with intended animation effects; (c)~a
set of built-in data structures for convenience; and (d)~a set of macros that
allows the program to traverse, for example, all incident edges of a node
without invoking templated Java constructs.
Line numbers of errors reported by the compiler are those of the code
displayed in the text window.
Runtime errors are reported in the same way.
In both cases, due to the imports and macro translations, the error messages
may not be immediately intelligible, but the line numbers \emph{are}
correctly identified.

% [Last modified: 2013 06 25 at 18:08:07 GMT]
