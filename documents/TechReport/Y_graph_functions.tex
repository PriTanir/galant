\begin{table}
  \centering
  \begin{tabular}{| l | p{0.6\textwidth} |}
    \hline
    \textsf{display(String s)}
    &
    writes the string \textsf{s} at the top of the window
    (not really a graph method, but currently associated with the graph)
    \\ \hline
    \textsf{String getMessage()}
    &
    returns the current message
    \\ \hline
    \textsf{NodeList getNodes()}
    &
    returns a list of the nodes of the graph; the return type is built into Galant
    and equivalent to the Java \textsf{List<Node>}
    \\ \hline
    \textsf{EdgeList getEdges()}
    &
    returns a list of edges of the graph; return type is analogous to \textsf{NodeList}
    \\ \hline
    \textsf{for\_nodes(node)}
    &
    equivalent to \textsf{for ( node : getNodes() )} and must be followed by a block
    of code enclosed in \textsf{\{ \}}'s
    \\ \hline
    \textsf{for\_edges(edge)}
    &
    analogous to \textsf{for\_nodes}
    \\ \hline
    \textsf{getStartNode()}
    &
    returns the first node in the list of nodes, typically the one with smallest id;
    used by algorithms that require a start node
    \\ \hline
    \textsf{Node addNode()}
    &
    returns a new node and adds it to the list of nodes;
    the id is one greater than the largest id so far;
    attributes such as weight, label and position are absent and must be set explicitly
    by appropriate method calls
    \\ \hline
    \textsf{addEdge(Node source, Node target)}
    &
    adds an edge from the source to
    the target (source and target are interchangeable when graph is undirected).
    There is also a variation with integer arguments that represent the id's of the nodes. As in the case of nodes, the edge is added to the list of edges and
    its weight and label are absent
    \\ \hline
  \end{tabular}
  \caption{Functions/methods and macros that apply to the whole graph.}
  \label{tab:graph_functions}
\end{table}
