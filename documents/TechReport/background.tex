Algorithm animation has a long history, dating back at least as far as the
work of Brown and Sedgewick~\cite{1988-Computer-Brown,1985-IEEE_Software-Brown}
and that of Bentley and Kernighan~\cite{1987-Animation-Bentley} in the 1980's.
The BALSA software, developed by Brown and Sedgewick, is a sophisticated system that provides several
elaborate examples of animations, including various balanced search trees,
Huffman trees, depth-first search, Dijkstra's algorithm and transitive closure.
The Bentley-Kernighan approach is simpler: an implementation of an algorithm is annotated with output directives that trace its execution.
These directives are later processed by an interpreter that
converts each directive into a still picture (or modification of a previous
picture). The pictures are then composed into a sequence that is navigated by the user.

In discussing algorithm animation software,
we distinguish three primary roles: the \emph{observer}, who simply
watches an animation; the \emph{explorer}, who interacts with an animation by,
for example, changing the problem instance (graph); and the \emph{animator}, who
designs an animation. The latter may also be referred to as a
\emph{developer} if the process of creating animations is integrated with
that of implementing the animation system.\footnote{
  With Galant a \emph{developer} is someone who participates in modifying the
  underlying Galant implementation, an activity completely independent of creating
  animations.
}
An explorer needs to be an observer as well
and an animator needs to be both of
the others.
R\"ossling and Freisleben~\cite{2002-JVLC-Roessling} articulate a similar
classification of roles.

We also define an algorithm animation \emph{tool} as animation software specifically designed to interact easily with other programs such as
text editors, other graph editors, other algorithm animation tools,
graph generators, Java API's, format translation filters and
graph drawing programs.

A hypothesis, at least partially validated (using student attitude
surveys~\cite{1997-SIGCSE-Stasko}) is
\begin{enumerate}
  \item \label{item:demonstration}
    The value added --- beyond lectures and textbook
    --- for students watching an animation (observer role)
    is minimal.
  \item \label{item:interactive}
    If students are able to manipulate problem instances (explorer role)
    the gain is more significant.
  \item \label{item:creative}
    Students who implement algorithms and design simple animations of them
    (animator role) are likely internalize the structure of the algorithm and
    come away with significant understanding of it.
\end{enumerate}

We call animations that are designed for (\ref{item:demonstration}) above, i.e., observer focused,
\emph{demonstration} animations;
those designed for (\ref{item:interactive}), i.e., for both observer and explorer, are
\emph{interactive} animations; those designed for (\ref{item:creative}) are
\emph{creative} animations.

% A hypothesis, at least partially validated (using student attitude
% surveys~\cite{1997-SIGCSE-Stasko}) posits that,
% while students acting as observers or explorers of animations may gain some benefit,
% a student who takes on the role of animator has a much richer learning
% experience.
% This hypothesis has become Galant's major motivating factor --- that it should
% simplify the task of the animator as much as possible while enhancing
% the ability to produce compelling animations.

% [Last modified: 2013 06 26 at 18:44:29 GMT]
