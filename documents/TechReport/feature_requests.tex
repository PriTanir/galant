Some features under development for the next major release are listed here.

\begin{enumerate}
\item \label{item:data_structures}
  Eventually
  lists, queues, stacks and priority queues of nodes and edges will all be
  concrete and all graph, node and edge methods that return containers will
  return one of these. For example, \texttt{getNodes()} will return a
  \texttt{NodeList} instead of a \texttt{List<Node>}.

\item
  \emph{Mode-less graph editing.}
  In place of the GDR-like mechanism, where panel buttons are used to determine
  how the graph editor responds to mouse actions -- a click might create a node,
  initiate an edge or select an object --
  the \textsf{Ctrl} and \textsf{Shift} keys can determine the ``mode''.
  So, for example, \textsf{Ctrl-Click} would create a node and
  \textsf{Shift-Click} would initiate an edge.

\item
  \emph{Mapping attributes to actions.}
  In order to make animations more accessible to visually impaired users,
  there should be a mechanism that, under user control, specifies how Boolean
  attributes such as marking or highlighting are ``displayed''.
  Currently, the thickness of highlighted node borders and edges can be
  controlled in the \textsf{Preferences} panel.
  A more sophisticated mapping mechanism that incorporates sound as well as visuals is needed.
  The ultimate approach would allow mappings for arbitrary attributes
  defined by user or programmer.

\item \emph{Inflection points on edges.}  For animation of automata it's
  important to have curved edges if there is a transition going from state
  $q$ to state $r$ and another from $r$ to $q$; other applications may need
  this as well. A single inflection point, carefully placed, and present only
  if there are parallel edges, could accomplish this.

\item \emph{Selection of multiple nodes and/or edges.}  It might prove useful
  to move a collection of nodes instead of just a single node or give a
  collection of nodes or edges the same color, label or weight. This is
  mostly for edit mode.
\end{enumerate}
