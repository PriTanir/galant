Animation programmers can write algorithms in notation that resembles
textbook pseudocode.
The animation examples have used procedural syntax for function calls, as in, for example,
\texttt{setWeight(v,0)}.
Java (object oriented) syntax can also be used: \texttt{v.setWeight(0)}.

The source code for an algorithm begins with any number (including none)
of global variable declarations and function definitions.
This is followed by the code for the algorithm itself, starting with the keyword
\textsf{algorithm}.
A \emph{code block}
is a sequence of statements, each terminated by a semicolon, just as in
Java.
The main algorithm has the form
\begin{quote}
\texttt{algorithm \{}\\
\hspace*{2em}\emph{code block}\\
\texttt{\}}
\end{quote}

Declarations of global variables are also like those of Java:\\
\hspace*{1em}\emph{type} \emph{variable\_name}\texttt{;}\\
or\\
\hspace*{1em}\emph{type} \texttt{[]} \emph{variable\_name}\texttt{;}\\
to declare an array of the given type.
All variables must be initialized either within a function definition or
in the algorithm.
The Java incantation\\
\hspace*{1em}\emph{type} \emph{variable\_name}
\texttt{= new} \emph{type}\texttt{[} \emph{size} \texttt{]}\\
is used to initialize an array with \emph{size} elements intialized to \textsf{null}
or 0.
Arrays use 0-based indexing: the largest index is $\mathit{size} - 1$.
Function declarations are described in more detail in Section~\ref{sec:functions}
below.

Central to the Galant API is the \texttt{graph} object: currently all other
parts of the API refer to it.
The components of a graph are declared to be of type \texttt{Node} or
\texttt{Edge} and can be accessed/modified via a variety of
functions/methods.
When an observer or explorer interacts with the animation they move either
forward or backward one step at a time.
All aspects of the graph API therefore refer to the current \emph{state of
  the graph}, the set of states behaving as a stack.
API calls that change the state of a node or an edge automatically
generate a next step,
but the programmer can override this using a \texttt{beginStep()} and
\texttt{endStep()} pair. For example, the beginning of our implementation of
Dijkstra's algorithm looks like
\begin{verbatim}
beginStep();
for_nodes(node) {
    setWeight(node, INFINITY);
    nodePQ.add(node);
}
endStep();
\end{verbatim}
Without the \texttt{beginStep}/\texttt{endStep}
override, this initialization would require the observer to click
through multiple steps (one for each node) before getting to the interesting
part of the animation.

\begin{table}
  \small
  \centering
  \begin{tabular}{| m{0.35\textwidth} | m{0.6\textwidth} |}
    \hline
    \shortstack[l]{
      \textsf{List$\langle$Node$\rangle$ getNodes()}\\
      \textsf{NodeList getNodes()}
    }
    &
    returns a list of the nodes of the graph; type \textsf{NodeList}
    is built into Galant
    and essentially equivalent to the Java \textsf{List$\langle$Node$\rangle$};
    see Table~\ref{tab:data_structures}
    \\ \hline
    \shortstack[l]{
      \textsf{List$\langle$Edge$\rangle$ edges()}\\
      \textsf{EdgeList edges()}
    }
    &
    returns a list of edges of the graph; return type
    \textsf{EdgeList} is analogous to \textsf{NodeList}
    \\ \hline
    \textsf{for\_nodes(v) \{
      \emph{statement; \ldots}
      \}}
    &
    \shortstack[l]{
      equivalent to
      \textsf{for ( Node v : nodes() ) \{ \emph{statement; \ldots} \}};\\
      the statements are executed for each node \textsf{v}
    }
    \\ \hline
    \textsf{for\_edges(e)  \{ \emph{statement; \ldots} \}}
    &
    analogous to \textsf{for\_nodes}
    \\ \hline
    \textsf{Integer numberOfNodes()}
    &
    returns the number of nodes
    \\ \hline
    \textsf{Integer numberOfEdges()}
    &
    returns the number of edges
    \\ \hline
    \textsf{int~id(Node~v)}, \textsf{int~id(Edge e)}
    &
    returns the unique identifier of \textsf{v} or \textsf{e}
    \\ \hline
    \textsf{int~nodeIds()}, \textsf{int~edgeIds()}
    &
    returns the largest node/edge identifier plus one;
    useful when an array is to be indexed using node/edge identifiers,
    since these are not necessarily contiguous
    \\ \hline
    \textsf{source(Edge e)}, \textsf{target(Edge e)}
    &
    returns the source/target of edge \textsf{e}, sometimes called the (arrow)
    tail/head or source/destination
    \\ \hline
    \shortstack[l]{
      \textsf{Integer degree(Node v)}\\
      \textsf{Integer indegree(Node v)}\\
      \textsf{Integer outdegree(Node v)}
    }
    &
    the number of edges incident on \textsf{v}, total, incoming and outgoing;
    if the graph is undirected, the outdegree is the same as the degree
    \\ \hline
    \shortstack[l]{
      \textsf{EdgeList edges(Node v)}\\
      \textsf{EdgeList inEdges(Node v)}\\
      \textsf{EdgeList outEdges(Node v)}
    }
    &
    returns a list of \textsf{v}'s
    indicent, incoming or outgoing edges, respectively;
    outgoing edges are the same as incident edges if the graph is undirected 
    \\ \hline
    \shortstack[l]{
      \textsf{Node otherEnd(Edge e, Node v)}\\
      \textsf{Node otherEnd(Node v, Edge e)}
    }
    &
    returns the node opposite \textsf{v} on edge \textsf{e};
    if \textsf{v} is the source \textsf{otherEnd} returns the target and
    vice-versa
    \\ \hline
    \textsf{NodeList neighbors(Node v)}
    &
    returns a list of nodes adjacent to \textsf{v}
    \\ \hline
    \shortstack[l]{
      \textsf{for\_adjacent(v, e, w) \{ \emph{code block} \}} \\
      \textsf{for\_incoming(v, e, w) \{ \emph{code block} \}} \\ 
      \textsf{for\_outgoing(v, e, w) \{ \emph{code block} \}}
    }
    &
    \textsf{for\_adjacent} executes the code block for each edge \textsf{e}
    incident on \textsf{v}, where \textsf{w} is \textsf{otherEnd(e,v)};
    \textsf{v} must already be declared but \textsf{e} and \textsf{w} are
    declared by the macro;
    the other two are analogous for incoming and outgoing edges 
    \\ \hline
    \textsf{getStartNode()}
    &
    returns the first node in the list of nodes, typically the one with smallest id;
    used by algorithms that require a start node
    \\ \hline
    \textsf{isDirected()}
    &
    returns true if the graph is directed
    \\ \hline
    \textsf{setDirected(boolean directed)}
    &
    makes the graph directed or undirected depending on whether \texttt{directed}
    is true or false, respectively
    \\ \hline
    \shortstack[l]{
      \textsf{Node addNode()}\\
      \textsf{Node addNode(Integer x, Integer y)}
    }
    &
    returns a new node and adds it to the list of nodes;
    the id is the smallest integer not currently in use as an id;
    attributes such as weight, label and position are absent and must be set explicitly
    by appropriate method calls;
    the second version sets the position of the node at \textsf{(x,y)}
    \\ \hline
    \shortstack[l]{
      \textsf{addEdge(Node source, Node target)}\\
      \textsf{addEdge(int sourceId, int targetId)}
    }
    &
    adds an edge from the source to
    the target (source and target are interchangeable when graph is undirected);
    the second variation specifies id's of the nodes to be connected;
    as in the case of adding a node, the edge is added to the list of edges and
    its weight and label are absent
    \\ \hline
    \textsf{deleteNode(Node v)}
    &
    removes node \textsf{v} and its incident edges from the graph
    \\ \hline
    \textsf{deleteEdge(Edge e)}
    &
    removes edge \textsf{e} from the graph
    \\ \hline
  \end{tabular}
  \caption{Functions and macros that apply to the structure of a graph.}
  \label{tab:graph_functions}
\end{table}


Functions and macros for the graph as a whole are shown in Table~\ref{tab:graph_functions}.
Also included are a few functions that are global to an algorithm without
any direct connection to a graph.

The nodes and edges, of type \textsf{Node} and \textsf{Edge}, respectively,
are subtypes of \textsf{GraphElement}.
Arbitrary attributes can be assigned to each graph element. In the GraphML file
these show up as, for example,\\
\hspace*{3em}
\textsf{
$<$node \emph{attribute\_1}="\emph{value\_1}" ... \emph{attribute\_k}="\emph{value\_k}" /$>$
}

Each node and edge has a unique integer id.
The id's are assigned consecutively as nodes/edges are created
and may not be altered.
The id of a node or edge can be accessed via the \texttt{id()} function.
Often, as is the case with the depth-first search algorithm, it makes sense to use
arrays indexed by node or edge id's.
Since graphs may be generated externally and/or have undergone deletions of nodes or
edges, the id's are not always contiguous.
The functions \texttt{nodeIds()} and \texttt{edgeIds()} return the size of an array
large enough to accomodate the appropriate id's as indexes. So code such as
\begin{verbatim}
Integer myArray[] = new Integer[nodeIds()];
for_nodes(v) {
    myArray[id(v)] = ...
}
\end{verbatim}
is immune to array out of bounds errors.

\subsection{Node and edge methods}

Nodes and edges have `getters' and `setters' for
a variety of attributes, i.e.,
\\
\texttt{set}$a$\texttt{($\langle a's$ type$\rangle$ x)}
\\
and
\\
$\langle a's$ type$\rangle$ \texttt{get}$a$\texttt{()},
where $a$ is the name of an attribute such as
\texttt{Color},
\texttt{Label} or \texttt{Weight}.
A more convenient way to access these standard attributes omits the prefix \texttt{get}
and uses procedural syntax:
\texttt{color(}$x$\texttt{)} is a synomym for $x$\texttt{.getColor()}, for example.
Procedural syntax for the setters is also available:
\texttt{setColor(}$x$,$c$\texttt{)} is a synonym for $x$.\texttt{setColor(}$c$\texttt{)}.
In the special cases of color and label it is possible to omit the \texttt{set}
(since it reads naturally): \texttt{color(}$x$,$c$\texttt{)} instead of \texttt{setColor(}$x$,$c$\texttt{)}.

\subsubsection{Logical attributes: functions and macros}

\textbf{Nodes.}
From a node's point of view we would like information about the adjacent nodes and incident edges.
The relevant \emph{methods} require the use of Java generics, but macros are provided
to simplify graph API access. The macros, which have equivalents in GDR, are:

\begin{itemize}

\item
\texttt{for\_adjacent(x, e, y)\{ \emph{statements} \}}
executes the list of statements for each edge incident on \verb$x$.
The statements can refer to \verb$x$, or \verb$e$, the current incident edge,
or \verb$y$, the other endpoint of \verb$e$.
The macro assumes that \texttt{x} has already been declared as \texttt{Node}
but \texttt{e} and \texttt{y} are declared automatically.

\item
\texttt{for\_outgoing(Node x, Edge e, Node y)\{ \emph{statements} \}}\\
behaves like \texttt{for\_adjacent} except that, when the graph is directed,
it iterates only over the edges whose source is \verb$x$ (it still iterates over all the edges when the graph is undirected). 

\item
\texttt{for\_incoming(Node x, Edge e, Node y)\{ \emph{statements} \}}\\
behaves like \texttt{for\_adjacent} except that, when the graph is directed,
it iterates only over the edges whose sink is \verb$x$ (it still iterates over all the edges when the graph is undirected). 

\end{itemize}

The actual API methods hiding behind these macros are (these are Node methods):

\begin{itemize}
\item
\texttt{List<Edge>~getIncidentEdges()} returns a list of all edges incident to this node,
both incoming and outgoing.
\item
\texttt{List<Edge>~getOutgoingEdges()} returns a list of edges directed away
from this node (all incident edges if the graph is undirected).
\item
\texttt{List<Edge>~getIncomingEdges()} returns a list of edges directed toward
this node (all incident edges if the graph is undirected).
\item
\texttt{Node~travel(Edge~e)} returns the other endpoint of \texttt{e}.
\end{itemize}

The above all use Java syntax, as in \texttt{v.travel(e)}.
The following are node-related functions with procedural syntax.

\begin{itemize}
\item \texttt{degree(v)}, \texttt{indegree(v)} and \texttt{outdegree(v)} return the appropriate
integers.
\item \texttt{otherEnd(v, e)}, where \texttt{v} is a node and \texttt{e} is an edge
returns node \texttt{w} such that \texttt{e} connects \texttt{v} and \texttt{w};
the programmer can also say \texttt{otherEnd(e, v)} in case she forgets the order
of the arguments.
\item \texttt{neighbors(v)} returns a list of the nodes adjacent to node \texttt{v}.
\end{itemize}

\bigskip
\textbf{Edges.}
The logical attributes of an edge are its source and target (destination).

\begin{itemize}
\item
\texttt{setSourceNode(Node)} and \texttt{Node~getSourceNode()}
\item
\texttt{setDestNode(Node)} and \texttt{Node~getDestNode()}
\item
\texttt{getOtherEndPoint(Node~u)} returns \texttt{v} where this edge is
either \texttt{uv} or \texttt{vu}.
\end{itemize}

\bigskip
\textbf{Graph Elements.}
Nodes and edges both have a mechanism for setting (and getting)
arbitrary attributes of type Integer, String, and Double.
the relevant methods are\\
\texttt{setIntegerAttribute(String~key,Integer~value)}\\ 
to associate an integer value with a node and\\
\texttt{Integer~getIntegerAttribute(String~key)}\\
to retrieve it.
String and Double attributes work the same way as integer attributes.
These are useful when an algorithm requires arbitrary information to be
associated with nodes and/or edges.
The user-defined attributes may differ from one node or edge to the next.
For example, some nodes may have a \texttt{depth} attribute while others do not.

\subsubsection{Geometric attributes}

Currently, the only geometric attributes are the positions of the
nodes. 
Unlike GDR, the edges in Galant
are all straight lines and the positions of their labels are fixed.
The relevant methods for nodes -- using procedural syntax -- are
\texttt{int~getX(Node)}, \texttt{int~getY(Node)}
and \texttt{Point~getPosition(Node)}
for the 'getters'. To set a position,
one should use either
\texttt{setPosition(Node,Point)}\\
or \texttt{setPosition(Node,int,int)}.
Once a node has an established position, it is possible to change
only one coordinate using \texttt{setX(Node,int)} or \texttt{setY(Node,int)}.
Object-oriented variants of all of these, e.g.,
\texttt{v.setX(100)}, are also available.

Ordinarily nodes can be moved by the user during algorithm execution
and the resulting positions persist after execution terminates.
For some algorithms, such as sorting, the algorithm itself needs to move
nodes.
It is desirable then to keep the user from moving nodes.
The declaration \texttt{movesNodes()} at the beginning of an algorithm
accomplishes this.

\subsubsection{Display attributes} \label{sec:display_attributes}

Each node and edge has
both a (double) weight and a label.
The weight
is also a logical
attribute in that
it is used implicitly as a
key for
sorting and priority queues.
The label is simply text and may be interpreted however the programmer
chooses.
Aside from the setters and getters: \texttt{setWeight(double)},
\mbox{\texttt{Double getWeight()}}, \texttt{setLabel(String)}
and \mbox{\texttt{String getLabel()}}, the programmer can also
manipulate and test for the absence of weights/labels using
\texttt{clearWeight()} and \texttt{boolean~hasWeight()},
and the corresponding methods for labels.
The procedural variants in this case are
\texttt{setWeight(Node,double)},
\mbox{\texttt{Double weight(Node)}},\footnote{
  The \emph{get} is omitted here for more natural syntax.}
\texttt{label(Node,String)},\footnote{
  A natural syntax that resembles English. However,
  \texttt{setLabel(Node,String)} is also allowed.
}
and \mbox{\texttt{String label(Node)}}

Nodes can either be plain, highlighted (selected), marked (visited) or both highlighted and
marked.
Being highlighted alters the
the boundary (color and thickness) of a node (as controlled by the
implementation), while being marked affects the fill color.
Edges can be plain or selected, with thickness and color modified in the
latter case.

The relevant methods are
(here \texttt{Element} refers to either a \texttt{Node} or an \texttt{Edge}):
\begin{itemize}
\item \texttt{highlight(Element)}, \texttt{unHighlight(Element)}
  and \texttt{Boolean~isHighlighted(Element)}
\item correspondingly, \texttt{setSelected(true)}, \texttt{setSelected(false)},
and \texttt{boolean~isSelected()}
\item \texttt{mark(Node)}, \texttt{unMark(Node)}
  and \texttt{Boolean~isMarked(Node)},
  equivalently \texttt{Boolean~marked(Node)}.
\end{itemize}

Although the specific colors for displaying selected nodes or edges are
predetermined, the animation implementation can modify the color of a node boundary
or an edge, thus allowing for many different kinds of highlighting.
The \texttt{setColor} and \texttt{getColor} methods use String arguments
using the RGB format \textsf{\#RRGGBB}; for example,
the string \texttt{\#0000ff} is blue.
There are several predefined color constants:

\begin{tabular}{l @{~} l}
\texttt{RED} & \texttt{"\#ff0000"} \\
\texttt{BLUE} & \texttt{"\#00ff00"} \\
\texttt{GREEN} & \texttt{"\#0000ff"} \\
\texttt{YELLOW} & \texttt{"\#ffff00"} \\
\texttt{MAGENTA} & \texttt{"\#ff00ff" } \\
\texttt{CYAN} & \texttt{"\#00ffff"} \\
\texttt{TEAL} & \texttt{"\#009999"} \\
\texttt{VIOLET} & \texttt{"\#9900cc"} \\
\texttt{ORANGE} & \texttt{"\#ff8000"} \\
\texttt{GRAY} & \texttt{"\#808080"} \\
\texttt{BLACK} & \texttt{"\#000000"} \\
\texttt{WHITE} & \texttt{"\#ffffff"}
\end{tabular}

Of the attributes listed above, weight, label, color and position can be
accessed and modified by the user as well as the program.
In all cases, modifications by execution of the animation are ephemeral
-- the graph returns to its original state after execution.

\begin{table}
  \small
  \centering
  \begin{tabular}{| m{0.4\textwidth} | m{0.55\textwidth} |}
    \hline
    \textsf{id(\emph{element})}
    &
    returns the unique identifier of the node or edge
    \\ \hline
    \textsf{mark(Node v), unmark(Node v)}
    &
    shades the interior of a node or undoes that
    \\ \hline
    \textsf{Boolean marked(Node v)}
    &
    returns \textsf{true} if the node is marked
    \\ \hline
    \textsf{highlight(\emph{element}), unhighlight(\emph{element})}
    &
    makes the node or edge highlighted, i.e.,
    thickens the border or line and makes it red / undoes the highlighting
    \\ \hline
    \textsf{Boolean highlighted(\emph{element})}
    &
    returns \textsf{true} if the node or edge is highlighted
    \\ \hline
    \shortstack[l]{
      \textsf{select(\emph{element})}, \textsf{deselect(\emph{element})}\\
      \textsf{selected(\emph{element})}
    }
    &
    synonyms for \textsf{highlight}, \textsf{unhighlight} and \textsf{highlighted}
    \\ \hline
    &
    \\ \hline
    \textsf{boolean set(\emph{element}, String key, $\langle$\emph{type}$\rangle$ value)}
    &
    sets an arbitrary attribute, \textsf{key}, of the element to have a value of a given type, where
    the type is one of \textsf{Integer}, \textsf{Double}, \textsf{Boolean}
    or \textsf{String};
    in the special case of \textsf{Boolean} the third argument may be omitted
    and defaults to \textsf{true};
    so \textsf{set(v,"attr")} is equivalent to \textsf{set(v,"attr",true)};
    returns \textsf{true} if the element already has a value for the given attribute,
    \textsf{false} otherwise
    \\ \hline
    \shortstack[l]{
    \textsf{$\langle$\emph{type}$\rangle$ get$\langle$\emph{type}$\rangle$(\emph{element}, String key)}\\
    \textsf{Boolean is(\emph{element}, String key)}
    }
    &
    returns the value associated with \textsf{key} or \textsf{null}
    if the graph has no value of the given type for \textsf{key}, i.e.,
    if no
    \textsf{set(String~key,~$\langle$\emph{type}$\rangle$~value)} has occurred;
    in the special case of a \textsf{Boolean} attribute, the second formulation
    may be used;
    the object-oriented syntax, such as \textsf{e.is("inTree")}, sometimes
    reads more naturally
    \\ \hline
  \end{tabular}

  \caption{Functions that query and manipulate attributes of individual
    nodes and edges.
    Here, \emph{element} refers to either a \textsf{Node} or an \textsf{Edge},
    both the type and the formal parameter.
  }
  \label{tab:graph_element_functions}
\end{table}


A summary of functions relevant to node and edge attributes (their procedural versions)
is given in Table~\ref{tab:graph_element_functions}.

\subsubsection{Global access for individual node/edge attributes and graph attributes}

\begin{table}
\centering

\small
\parbox{0.9\textwidth}{
  These functions are designed to access or manipulate attributes for all nodes or edges
  at once instead of individually.
  Also included are functions that deal with graph attributes.
}

\medskip
  \begin{tabular}{| m{0.37\textwidth} | m{0.58\textwidth} |}
    \hline
    \shortstack[l]{
      \Code{Boolean nodeLabelsAreVisible()}\\
      \Code{Boolean edgeLabelsAreVisible()}\\
      \Code{Boolean nodeWeightsAreVisible()}\\
      \Code{Boolean edgeWeightsAreVisible()}
    }
    &
    returns \Code{true} if node/edge labels/weights are \emph{globally} visible,
    the default state, which can be altered by \Code{hideNodeLabels()}, etc.,
    defined below
    \\ \hline
    \shortstack[l]{
      \Code{hideNodeLabels(), hideEdgeLabels()}\\
      \Code{hideNodeWeights(), hideEdgeWeights()}
    }
    &
    hides all node/edge labels/weights; typically used at the beginning of an algorithm
    to hide unnecessary information; labels and weights are shown by default
    \\ \hline
    \shortstack[l]{
      \Code{showNodeLabels(), showEdgeLabels()}\\
      \Code{showNodeWeights(), showEdgeWeights()}
    }
    &
    undoes the hiding of labels/weights
    \\ \hline
    \shortstack[l]{
      \Code{hideAllNodeLabels()}\\
      \Code{hideAllEdgeLabels()}\\
      \Code{hideAllNodeWeights()}\\
      \Code{hideAllEdgeWeights()}
    }
    &
    hides all node/edge labels/weights even if they are visible globally
    by default or by
    \Code{showNodeLabels()}, etc., or for individual nodes and edges;
    in order for the label or weight of a node/edge to be displayed,
    labels/weights must be visible globally and its label/weight must be visible;
    initially, all labels/weights are visible, both globally and for individual
    nodes/edges;
    these functions are used to hide information so that it can be revealed subsequently,
    one node or edge at a time
    \\ \hline
    \shortstack[l]{
      \Code{showAllNodeLabels()}\\
      \Code{showAllEdgeLabels()}\\
      \Code{showAllNodeWeights()}\\
      \Code{showAllEdgeWeights()}
    }
    &
    makes all individual
    node/edge weights/labels visible if they are globally visible by default
    or via \Code{showNodeLabels()}, etc.;
    this undoes the effect of \Code{hideAllNodeLabels()}, etc., and of
    any individual hiding of labels/weights
    \\ \hline
    \shortstack[l]{
      \Code{clearNodeLabels(), clearEdgeLabels()}\\
      \Code{clearNodeWeights(), clearEdgeWeights()}
    }
    &
    gets rid of all node/edge labels/weights; this not only makes them invisible,
    but also erases whatever values they have
    \\ \hline
    \Code{showNodes(), showEdges()}
    &
    undo any hiding of nodes/edges that has taken place during the algorithm
    \\ \hline
    \shortstack[l]{
      \Code{NodeSet visibleNodes()} \\
      \Code{EdgeSet visibleEdges()}
      }
    &
    return the set of nodes/edges that are not hidden
    \\ \hline
    \shortstack[l]{
      \Code{clearNodeMarks()}\\
      \Code{clearNodeHighlighting()}\\
      \Code{clearEdgeHighlighting()}
    }
    &
    unmarks all nodes, unhighlights all nodes/edges, respectively
    \\ \hline
    \shortstack[l]{
      \Code{clearNodeLabels()}\\
      \Code{clearNodeWeights()}\\
      \Code{clearEdgeLabels()}\\
      \Code{clearEdgeWeights()}
    }
    &
    erases labels/weights of all nodes/edges; useful if an algorithm needs to
    start with a clean slate with respect to any of these attributes
    \\ \hline
    \shortstack[l]{
      \Code{clearAllNode(String attribute)}\\
      \Code{clearAllEdge(String attribute)}
    }
    &
    erases values of the given attribute from all nodes/edges, a generalization of
    \Code{clearNodeLabels}, etc.
    \\ \hline
    \Code{boolean set(String attribute, $\langle$\emph{type}$\rangle$ value)}
    &
    sets an arbitrary attribute of the graph to have a value of a given type, where
    the type is one of \Code{Integer}, \Code{Double}, \Code{Boolean}
    or \Code{String};
    in the special case of \Code{Boolean} the second argument may be omitted
    and defaults to \Code{true};
    so \Code{set("attr")} is equivalent to \Code{set("attr",true)};
    returns \Code{true} if the graph already has a value for the given attribute,
    \Code{false} otherwise
    \\ \hline
    \shortstack[l]{
    \Code{$\langle$\emph{type}$\rangle$ get$\langle$\emph{type}$\rangle$(String attribute)}\\
    \Code{Boolean is(String attribute)}
    }
    &
    returns the value associated with \Code{attribute} or \Code{null}
    if the graph has no value of the given type for \Code{attribute}, i.e.,
    if no
    \Code{set(String~attribute,~$\langle$\emph{type}$\rangle$~value)} has occurred;
    in the special case of a \Code{Boolean} attribute, the second formulation
    may be used
    \\ \hline
    \shortstack[l]{
      \Code{clearAllNode(String attribute)}\\
      \Code{clearAllEdge(String attribute)}
    }
    &
    erases the value of the given attribute for all nodes/edges
    \\ \hline
  \end{tabular}

  \caption{Functions that query and manipulate graph
    node and edge attributes globally.}
  \label{tab:graph_attribute_functions}
\end{table}

% [Last modified: 2017 01 17 at 18:39:12 GMT]


It is sometimes useful to access or manipulate attributes of nodes and edges
globally.
For example, an algorithm might want to hide node weights entirely
because they are not relevant
or hide them initially and reveal them for individual nodes as
the algorithm progresses.
These functionalities can be accomplished by
\textsf{hideNodeWeights} or \textsf{hideAllNodeWeights}, respectively.
A summary of these capabilities is given in Table~\ref{tab:graph_attribute_functions}.

\subsection{Additional programmer information}

A Galant algorithm/program is executed as a method within a Java class.
In order to shield the Galant programmer from Java ideosyncrasies,
some features have been added.

\subsubsection{Definition of Functions/Methods}\label{sec:functions}

A programmer can define arbitrary functions (methods) using the construct

\texttt{function} \textsl{[return\_type]} \textsl{name} \texttt{(}
 \textsl{parameters} \texttt{) \{} \\
 \hspace*{3em} \emph{code block} \\
 \texttt{\}}

The behavior is like that of a Java method. So, for example,
\begin{verbatim}
function int plus( int a, int b ) {
    return a + b;
}
\end{verbatim}
is equivalent to
\begin{verbatim}
static int plus( int a, int b ) {
    return a + b;
}
\end{verbatim}

The \textsl{return\_type} is optional. If it is missing, the function behaves like
a \textsf{void} method in Java. An example is the recursive
\\
\texttt{function visit( Node v ) \{} \textsl{code} \texttt{\}}
\\
The conversion of functions into Java methods when Galant code is compiled
is complex and may result in indecipherable error messages.

\begin{table}
  \small
  \centering

  \textbf{NodeList and EdgeList:} lists of nodes or edges, respectively.
  
  \medskip
  \begin{tabular}{| m{0.35\textwidth} | m{0.6\textwidth} |}
    \hline
    \shortstack[l]{
      \textsf{Node first(NodeList L)}\\
      \textsf{Edge first(EdgeList L)}
    }
    &
    returns the first element of this list
    \\ \hline
    \shortstack[l]{
      \textsf{add(Node v, NodeList L)}\\
      \textsf{add(Edge e, EdgeList L)}
    }
    &
    adds the node/edge to the end of the list \textsf{L}; along with \textsf{first}
    we get the effect of a queue
    \\ \hline
    \shortstack[l]{
      \textsf{remove(Node v, NodeList L)}\\
      \textsf{remove(Edge e, EdgeList L)}
    }
    &
    removes the first occurrence of node \textsf{v} or edge \textsf{e} from \textsf{L}
    \\ \hline
    \textsf{sort(NodeList L)}, \textsf{sort(EdgeList L)}
    &
    use the weights of the nodes/edges to sort the list \textsf{L}
    \\ \hline
  \end{tabular}  
  
\bigskip
  \textbf{NodeQueue and EdgeQueue:} queues of nodes or edges, respectively.

  \medskip
  \begin{tabular}{| m{0.35\textwidth} | m{0.6\textwidth} |}
    \hline
    \shortstack[l]{
      \textsf{void~enqueue(Node~v)},\\
      \textsf{void~enqueue(Edge~e)}
    }
    &
    adds \textsf{v} or \textsf{e} to the rear of the queue
    \\ \hline
    \textsf{Node~dequeue()}, \textsf{Edge~dequeue()}
    &
    returns and removes the \textsf{Node} or \textsf{Edge} at the front of the queue;
    returns \textsf{null} if the queue is empty
    \\ \hline
    \textsf{Node~remove()}, \textsf{Edge~remove()}
    &
    returns and removes the \textsf{Node} or \textsf{Edge} at the front of the queue;
    throws an exception if the queue is empty
    \\ \hline
    \textsf{Node~element()}, \textsf{Edge~element()}
    &
    returns the \textsf{Node} or \textsf{Edge} at the front of the queue
    without removing it;
    throws an exception if the queue is empty
    \\ \hline
    \textsf{Node~peek()}, \textsf{Edge~peek()}
    &
    returns the \textsf{Node} or \textsf{Edge} at the front of the queue
    without removing it;
    returns \textsf{null} if the queue is empty
    \\ \hline
    \textsf{size()}
    &
    returns the number of elements in the queue
    \\ \hline
    \textsf{isEmpty()}
    &
    returns \textsf{true} if the queue is empty 
    \\ \hline
  \end{tabular}

  \bigskip
  \textbf{NodeStack and EdgeStack:} stacks of nodes or edges, respectively.

  \medskip
  \begin{tabular}{| m{0.35\textwidth} | m{0.6\textwidth} |}
    \hline
    \shortstack[l]{
      \textsf{void~push(Node~v)},\\
      \textsf{void~push(Edge~e)}
    }
    &
    adds \textsf{v} or \textsf{e} to the top of the stack
    \\ \hline
    \textsf{Node~pop()}, \textsf{Edge~pop()}
    &
    returns and removes the \textsf{Node} or \textsf{Edge} at the top of the stack;
    throws an exception if the stack is empty
    \\ \hline
    \textsf{Node~peek()}, \textsf{Edge~peek()}
    &
    returns the \textsf{Node} or \textsf{Edge} at the top of the stack
    without removing it;
    returns null if the stack is empty
    \\ \hline
    \textsf{size()}, \textsf{isEmpty()}
    &
    analogous to the corresponding queue methods
    \\ \hline
  \end{tabular}

  \bigskip
  \textbf{NodePriorityQueue and EdgePriorityQueue:} priority queues of nodes or edges, respectively.

  \medskip
  \begin{tabular}{| m{0.35\textwidth} | m{0.6\textwidth} |}
    \hline
    \shortstack[l]{
      \textsf{void~add(Node~v)},\\
      \textsf{void~add(Edge~e)}
    }
    &
    adds \textsf{v} or \textsf{e} to the priority queue;
    the priority is defined to be its weight -- see Section~\ref{sec:display_attributes}
    \\ \hline
    \textsf{Node~removeMin()}, \textsf{Edge~removeMin()}
    &
    returns and removes the \textsf{Node} or \textsf{Edge} with minimum weight;
    returns \textsf{null} if the queue is empty
    \\ \hline
    \hline
    \shortstack[l]{
      \textsf{boolean~remove(Node~v)},\\
      \textsf{boolean~remove(Edge~e)}
    }
    &
    removes \textsf{v} or \textsf{e};
    returns \textsf{true} if the \textsf{v} or \textsf{e} is present,
    \textsf{false} otherwise
    \\ \hline
     \shortstack[l]{
      \textsf{void~decreaseKey(Node~v,~double~key)},\\
      \textsf{void~decreaseKay(Edge~e,~double~key)}
    }
    &
    changes the weight of \textsf{v} or \textsf{e} to the new key
    and reorganizes the priority queue appropriately;
    since this is accomplished by removing and reinserting the object, i.e.,
    inefficiently, this method can also be used to increase the key
    \\ \hline
    \textsf{size()}, \textsf{isEmpty()}
    &
    analogous to the corresponding queue methods
    \\ \hline
  \end{tabular}

  \caption{Built-in data structures and their methods.
     These methods use object-oriented syntax:
$\langle$\emph{structure}$\rangle$.$\langle$\emph{method}$\rangle$($\langle$\emph{arguments}$\rangle$)
    and are created using, e.g.,
    \textsf{NodeQueue~Q~=~new~NodeQueue();} the \textsf{new} operator in Java.
  }
  \label{tab:data_structures}
\end{table}


\subsubsection{Data Structures}

Galant provides some standard data strutures for nodes and edges automatically.
These are described in detail in Table~\ref{tab:data_structures}.

